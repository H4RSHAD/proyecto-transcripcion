

\newcommand{\jira}{img/jira.png}
\newcommand{\architec}{img/arquitec.png}
\newcommand{\github}{img/github.png}
\newcommand{\project}{img/project.png}
\newcommand{\python}{img/python.png}
\newcommand{\javas}{img/javas.png}
\newcommand{\flask}{img/flask.png}
\newcommand{\html}{img/html.png}
\newcommand{\css}{img/css.png}
\newcommand{\tailwind}{img/tailwind.png}
\newcommand{\postgre}{img/postgre.png}
\newcommand{\aws}{img/aws.png}
\newcommand{\traslate}{img/traslate.png}
\begin{doublespace}     %doble espaciado tailwind
    %==================================================================================================
    \section{METODOLOGIA SCRUM}
    Scrum es un proceso de gestión que reduce la complejidad en el desarrollo de productos
    para satisfacer las necesidades de los clientes. En el cual, se aplican de manera regular un
    “conjunto de buenas prácticas para trabajar colaborativamente, en equipo, y obtener el mejor
    resultado posible de un proyecto”. \par
    Scrum es un marco de trabajo ágil que permite a los equipos de desarrollo entregar productos
    funcionales en ciclos cortos de trabajo, llamados Sprints. Cada Sprint tiene una duración
    fija, generalmente de 2 a 4 semanas, y al final de cada Sprint se entrega un incremento del
    producto que puede ser revisado y evaluado por el cliente. \par
    El marco de trabajo Scrum se basa en tres pilares fundamentales: la transparencia, la
    inspección y la adaptación. Estos pilares permiten a los equipos de desarrollo trabajar de
    manera colaborativa y adaptarse a los cambios en los requisitos del cliente. \par
    Scrum se basa en la idea de que los equipos de desarrollo son autoorganizados y
    multifuncionales, lo que significa que cada miembro del equipo tiene habilidades y
    conocimientos diversos que les permiten trabajar juntos para lograr un objetivo común. \par
    El marco de trabajo Scrum se compone de roles, eventos y artefactos. Los roles incluyen el
    Product Owner, el Scrum Master y el equipo de desarrollo. Los eventos incluyen la
    planificación del Sprint, la reunión diaria, la revisión del Sprint y la retrospectiva del
    Sprint. Los artefactos incluyen el Product Backlog, el Sprint Backlog y el incremento del
    producto. \par
    En resumen, Scrum es un marco de trabajo ágil que permite a los equipos de desarrollo
    trabajar de manera colaborativa y adaptarse a los cambios en los requisitos del cliente. Se
    basa en la idea de que los equipos de desarrollo son autoorganizados y multifuncionales,
    y se compone de roles, eventos y artefactos que permiten a los equipos trabajar de manera
    efectiva y eficiente. \par



    
    %==================================================================================================
\clearpage  %nueva pagina
    
    \subsection{SPRINT N° 0}
    \subsubsection{ACTIVIDADES DE DEFINICION INICIAL (R-1)}
    El objetivo principal del “Spint 0” es preparar el entorno y sentar las bases para el desarrollo del
proyecto en los siguientes Sprint y es conveniente realizarlo para: Proyectos complejos, Equipos
nuevos, Cambios tecnológicos significativos \par

    % Definición de colores personalizados
    \definecolor{lightgray}{rgb}{0.9, 0.9, 0.9}
    \definecolor{lightblue}{rgb}{0.8, 0.9, 1.0}
    \definecolor{aliceblue}{rgb}{0.94, 0.97, 1.0}
    \definecolor{bleudefrance}{rgb}{0.19, 0.55, 0.91}

    

    \renewcommand{\arraystretch}{1.5} % Espaciado entre filas
    %\setlength{\tabcolsep}{5pt} % Espaciado entre columnas

    \begin{longtable}{|c|p{15cm}|}
    %\caption{Definiciones iniciales - Sprint 0 (opcional)}\\
    \hline
    \rowcolor{bleudefrance}

    \multicolumn{2}{c|}{\color{aliceblue}\Large\textbf{Definiciones iniciales - Sprint 0 (opcional)}}\\
    \hline
    \rowcolor{bleudefrance} \color{aliceblue}{\Large\textbf{Id}} & \color{aliceblue}\Large\textbf{\centering Tarea}\\
    \hline
    \endfirsthead
    
    %Esto es para que la tabla se repita en cada pagina, cuando es muy larga
    \rowcolor{bleudefrance}
    \hline 
    \multicolumn{2}{c|}{\color{aliceblue}\textbf{Definiciones iniciales - Sprint 0 (opcional)} (continuación)}\\
    \hline
    \rowcolor{bleudefrance} \color{aliceblue}{\Large\textbf{Id}} & \color{aliceblue}\textbf{\centering Tarea}\\
    \hline
    \endhead
    
    a & Definir el equipo Scrum, multifuncional y auto gestionado \\
    \rowcolor{lightblue} b & Definir el objetivo del producto\\
    \hline
    c & Identificar los requerimientos iniciales \\
    \hline
    \rowcolor{lightblue} d & Definir tiempo de duración de Sprint (ideal el mismo tiempo\par
        para todos de 2 a 4 semanas)\\
    \hline
    e & Definir infraestructura tecnológica (hardware y software)
     para la Gestión del proyecto (Jira , Azure DevOps y similares ) como también para el proceso de
    desarrollo del software\\
    \hline
    \rowcolor{lightblue} f & Definir patrón de desarrollo que definan actividades técnicas de desarrollo que
    serán realizadas durante el sprint \\
    \hline
    g & Esbozar modelos inicial de: Contexto, de Datos y Arquitectura\\
    \hline
    \rowcolor{lightblue} h & Definición criterios de calidad\\
    \hline
    i & Generar el Product Backlog (artefacto) , ver ejemplo de formato F3\\
    \hline
    \rowcolor{bleudefrance} \multicolumn{2}{c|}{} \\
    \hline
    
    \end{longtable}

\clearpage  %nueva pagina
\begin{enumerate}[label=\alph*)]
    \item \textbf{Descripcion del equipo SCRUM, multifuncionales y autogestionados con sus respectivas}
    
       % Definición de colores personalizados
       \definecolor{lightgray}{rgb}{0.9, 0.9, 0.9}
       \definecolor{lightblue}{rgb}{0.8, 0.9, 1.0}
       \definecolor{aliceblue}{rgb}{0.94, 0.97, 1.0}
       \definecolor{bleudefrance}{rgb}{0.19, 0.55, 0.91}
   
       
   
       \renewcommand{\arraystretch}{1.1} % Espaciado entre filas
       %\setlength{\tabcolsep}{5pt} % Espaciado entre columnas
   
       %Scrum Master
       \begin{longtable}{|p{4cm}|p{5cm}|p{6cm}|}
       %\caption{Definiciones iniciales - Sprint 0 (opcional)}\\
       \hline
       \rowcolor{bleudefrance}
   
       \multicolumn{3}{c|}{\color{aliceblue}\Large\textbf{ROL: SCRUM MÁSTER}}\\
       \hline
       \rowcolor{bleudefrance} \color{aliceblue}{ \textbf{Responsable}} & \color{aliceblue}\textbf{Características} & \color{aliceblue}\textbf{Justificación}\\
       \hline
       \endfirsthead
       
       %Esto es para que la tabla se repita en cada pagina, cuando es muy larga
       \rowcolor{bleudefrance}
       \hline 
       \multicolumn{2}{c|}{\color{aliceblue}\textbf{ROL: SCRUM MÁSTER} (continuación)}\\
       \hline
       \rowcolor{bleudefrance} \color{aliceblue}{\textbf{Responsable}} & \color{aliceblue}\textbf{Características} & \color{aliceblue}\textbf{Justificación}\\
       \hline
       \endhead
        % Aquí centramos "Franz" en ambas direcciones
        %\multicolumn{1}{|c|}{\vfill\centering \textbf{Franz}\vfill}
       {\vfill\centering \textbf{RIBERA SAAVEDRA FRANZ LEONARDO}\vfill} & Es una persona muy
       responsable y comprometida
       con su trabajo, mantiene
       buena comunicación y
       coordinación con todo el
       grupo. Escucha opiniones de
       todos y nos ayuda en las
       dudas que tengamos & 
       Tiene más experiencia en el
       desarrollo de sistemas,
       actualiza sus conocimientos
       sobre las herramientas a usar
       en el proyecto, conoce bien
       el trabajo de la metodología
       de Scrum y lo aplica de
       manera correcta, ayuda de
       manera equitativa al equipo,
       tiene buena comunicación y
       coordinación con el equipo\\
       \hline
       \rowcolor{bleudefrance} \multicolumn{3}{c|}{} \\
       \hline
       
       \end{longtable}
   
    %Product Owner
       
    \begin{longtable}{|p{4cm}|p{6cm}|p{6cm}|}
        %\caption{Definiciones iniciales - Sprint 0 (opcional)}\\
        \hline
        \rowcolor{bleudefrance}
    
        \multicolumn{3}{c|}{\color{aliceblue}\Large\textbf{ROL: PRODUCT OWNER}}\\
        \hline
        \rowcolor{bleudefrance} { \textbf{\color{aliceblue} Responsable}} & \textbf{\color{aliceblue} Características} & \textbf{\color{aliceblue} Justificación}\\
        \hline
        \endfirsthead
        
        %Esto es para que la tabla se repita en cada pagina, cuando es muy larga
        \rowcolor{bleudefrance}
        \hline 
        \multicolumn{2}{c|}{\color{aliceblue}\Large\textbf{ROL: PRODUCT OWNER} (continuación)}\\
        \hline
        \rowcolor{bleudefrance} \color{aliceblue}{ \textbf{Responsable}} & \color{aliceblue}\textbf{Características} & \color{aliceblue}\textbf{Justificación}\\
        \hline
        \endhead
         % Aquí centramos "Franz" en ambas direcciones
         %\multicolumn{1}{|c|}{\vfill\centering \textbf{Franz}\vfill}
        {\vfill\centering \textbf{GUZMAN HONOR WILSON}\vfill} & Es una persona responsable y
        comprometida con el equipo
        de desarrollo, demuestra
        predisposición para aclarar
        cualquier duda acerca del
        modelo de negocio de la
        empresa y además tiene la
        capacidad necesaria para la
        toma de decisiones. & 
        Tiene el conocimiento
        respecto al modelo de
        negocio, puede facilitar entre
        los requisitos funcionales o
        no funcionales al momento
        de añadir al Product backlog,
        además conoce el flujo de7
        actividades respecto al
        servicio técnico y tiene la
        facilidad de negociar con el
        cliente del producto.\\
        \hline
        \rowcolor{bleudefrance} \multicolumn{3}{c|}{} \\
        \hline
        
        \end{longtable}


        %Desarrolladores

        \begin{longtable}{|p{7cm}|p{4cm}|p{5cm}|}
            %\caption{Definiciones iniciales - Sprint 0 (opcional)}\\
            \hline
            \rowcolor{bleudefrance}
        
            \multicolumn{3}{c|}{\color{aliceblue}\Large\textbf{ROL: TEAM DEVELOPER}}\\
            \hline
            \rowcolor{bleudefrance} \color{aliceblue}{ \textbf{Responsable}} & \color{aliceblue}\textbf{Características} & \color{aliceblue}\textbf{Justificación}\\
            \hline
            \endfirsthead
            
            %Esto es para que la tabla se repita en cada pagina, cuando es muy larga
            \rowcolor{bleudefrance}
            \hline 
            \multicolumn{2}{c|}{\color{aliceblue}\textbf{ROL: TEAM DEVELOPER} (continuación)}\\
            \hline
            \rowcolor{bleudefrance} \color{aliceblue}{ \textbf{Responsable}} & \color{aliceblue}\textbf{Características} & \color{aliceblue}\textbf{Justificación}\\
            \hline
            \endhead
             % Aquí centramos "Franz" en ambas direcciones
             %\multicolumn{1}{|c|}{\vfill\centering \textbf{Franz}\vfill}
            {\vfill\centering \textbf{MONTALVAN MEDINA KAREN}\par 
            \textbf{RODRIGUEZ VALENCIA JOSE LUIS}\par 
            \textbf{SAGREDO CORDOVA SIMON}      \par 
            \textbf{SOSA PESOA JUAN ANTONIO}    \par \vfill} & 
            Personas comprometida con
            su trabajo designado,
            colaborativo y participativo
            en las reuniones. & 
            Pertenece al team developer.
            Actualiza sus conocimientos
            sobre las herramientas en el
            proyecto, tiene gran
            conocimiento sobre la
            metodología de scrum.\\
            \hline
            \rowcolor{bleudefrance} \multicolumn{3}{c|}{} \\
            \hline
            
            \end{longtable}
    \item \textbf{Definir el objetivo del producto}\par
    El objetivo del producto es la creación de un software que permita la transcripción y/o
    traducción de conferencias presenciales, videos y audios con inteligencia artificial.


    \item \textbf{Identificar los requerimientos iniciales}
    \begin{itemize}
        \item Requerimientos funcionales
        \begin{itemize}
            \item RF-1: El sistema debe permitir la transcripción de conferencias presenciales.
            \item RF-2: El sistema debe permitir la transcripción de videos.
            \item RF-3: El sistema debe permitir la transcripción de audios.
            \item RF-4: El sistema debe permitir la traducción de conferencias presenciales a otro idioma.
            \item RF-5: El sistema debe permitir la traducción de videos a cualquier idioma que el usuaro eliga.
            \item RF-6: El sistema debe permitir la traducción de audios a cualquier idioma que el usuaro eliga.
        \end{itemize}
        
    \end{itemize}
        
    \item \textbf{Definir tiempo de duración de Sprint (ideal el mismo tiempo
    para todos de 2 a 4 semanas)}

    %Tiempo de duración de Sprint 
       
    \begin{longtable}{|c|c|c|c|c|}
        %\caption{Definiciones iniciales - Sprint 0 (opcional)}\\
        \hline
        \rowcolor{bleudefrance}
    
        \multicolumn{5}{c|}{\color{aliceblue}\Large\textbf{Definir tiempo de duración de Sprint}}\\
        \hline
        \rowcolor{bleudefrance} \color{aliceblue}{ \textbf{NRO}} & \color{aliceblue}\textbf{ACTIVIDAD} & \color{aliceblue}\textbf{FECHA INICIO} & \color{aliceblue}\textbf{FECHA FINALIZACION} & \color{aliceblue}\textbf{DURACION DIAS}\\
        \hline
        \endfirsthead
        
        %Esto es para que la tabla se repita en cada pagina, cuando es muy larga
        \rowcolor{bleudefrance}
        \hline 
        \rowcolor{bleudefrance} \color{aliceblue}{ \textbf{NRO}} & \color{aliceblue}\textbf{ACTIVIDAD} & \color{aliceblue}\textbf{FECHA INICIO} & \color{aliceblue}\textbf{FECHA FINALIZACION} & \color{aliceblue}\textbf{DURACION DIAS}\\
        \hline
        \endhead

        {\textbf{1}} & SPRINT 0 & 07/MAYO/2025 & 10/MAYO/2025 & 3 \\
        \hline
        \rowcolor{lightblue}{\textbf{2}} & SPRINT 1 & 12/MAYO/2025 & 25/MAYO/2025 & 15 \\
        \hline
        {\textbf{3}} & SPRINT 2 & 26/MAYO/2025 & 08/JUNIO/2025 & 15 \\
        \hline
        \rowcolor{lightblue}{\textbf{4}} & SPRINT 3 & 09/JUNIO/2025 & 22/JUNIO/2025 & 15 \\
        \hline
        {\textbf{5}} & SPRINT 4 & 23/JUNIO/2025 & 30/JUNIO/2025 & 8 \\
        \hline
        \rowcolor{bleudefrance} \multicolumn{5}{c|}{} \\
        \hline
        
        \end{longtable}
    
    \item \textbf{Definir infraestructura tecnológica (hardware y software) para la Gestión del proyecto}
    

        %Herramienta de gestion 
        \begin{longtable}{|p{3cm}|p{6cm}|p{6cm}|}
            %\caption{Definiciones iniciales - Sprint 0 (opcional)}\\
            \hline
            \rowcolor{bleudefrance}
        
            \multicolumn{3}{c|}{\color{aliceblue}\Large\textbf{Software para la Gestión de Proyectos: Jira Software }}\\
            \hline
            \rowcolor{bleudefrance} \color{aliceblue}{ \textbf{Logo}} & \color{aliceblue}\textbf{CARACTERISTICA} & \color{aliceblue}\textbf{JUSTIFICACION} \\
            \hline
            \endfirsthead
            
            %Esto es para que la tabla se repita en cada pagina, cuando es muy larga
            \rowcolor{bleudefrance}
            \hline 
            \rowcolor{bleudefrance} \color{aliceblue}{ \textbf{Logo}} & \color{aliceblue}\textbf{CARACTERISTICA} & \color{aliceblue}\textbf{JUSTIFICACION} \\           
            \hline
            \endhead
    
% Fila de datos \includegraphics[width=\linewidth]{\logoqr}
    \raisebox{-\totalheight}{\includegraphics[width=3cm]{\jira}} & 
    \begin{itemize}
        \item Herramienta visual para gestionar proyectos ágiles.
        \item Visualización de métricas de rendimiento del equipo.
        \item Compatibilidad con Git, Bitbucket, Confluence, entre otros.
        \item Seguimiento y gestión de versiones de software.
    \end{itemize} & 
    Este software nos colabora en los eventos scrum (sprint planning, daily sprint,
    retrospective y sprint review). Además, un proyecto de Jira Software es altamente configurable y
    se puede personalizar fácilmente para adaptarse a la estructura organizativa, al flujo de trabajo o
    al nivel de madurez en metodología ágil. \\
    \hline

            \hline
            \rowcolor{bleudefrance} \multicolumn{3}{c|}{} \\
            \hline
            
            \end{longtable}
%----------------------------------

        % Herramienta de Modelado
       
        \begin{longtable}{|p{3cm}|p{6cm}|p{6cm}|}
            %\caption{Definiciones iniciales - Sprint 0 (opcional)}\\
            \hline
            \rowcolor{bleudefrance}
        
            \multicolumn{3}{c|}{\color{aliceblue}\Large\textbf{Software de Modelado UML: ENTERPRISE ARCHITECT}}\\
            \hline
            \rowcolor{bleudefrance} \color{aliceblue}{ \textbf{Logo}} & \color{aliceblue}\textbf{CARACTERISTICA} & \color{aliceblue}\textbf{JUSTIFICACION} \\
            \hline
            \endfirsthead
            
            %Esto es para que la tabla se repita en cada pagina, cuando es muy larga
            \rowcolor{bleudefrance}
            \hline 
            \rowcolor{bleudefrance} \color{aliceblue}{ \textbf{Logo}} & \color{aliceblue}\textbf{CARACTERISTICA} & \color{aliceblue}\textbf{JUSTIFICACION} \\           
            \hline
            \endhead
    
% Fila de datos \includegraphics[width=\linewidth]{\logoqr}
    \raisebox{-\totalheight}{\includegraphics[width=3cm]{\architec}} & 
    \begin{itemize}
        \item Útil para visualizar, analizar, modelar, probar y mantener todos sus sistemas,
        software, procesos y arquitecturas.
        \item Seguimiento de la implementación de los requisitos del sistema para modelar
        elementos
        \item Buscar e informar sobre los requisitos
    \end{itemize} & 
    Esta herramienta fue escogida porque cumple todos los requisitos para el diseño UML, a
la misma vez el equipo de desarrollo conoce la herramienta casi en su totalidad, lo cual nos
ahorrará tiempo con un software que se conoce. \\
    \hline

            \hline
            \rowcolor{bleudefrance} \multicolumn{3}{c|}{} \\
            \hline
            
            \end{longtable}
%----------------------------------




        % Herramienta de Modelado
       
        \begin{longtable}{|p{3cm}|p{6cm}|p{6cm}|}
            %\caption{Definiciones iniciales - Sprint 0 (opcional)}\\
            \hline
            \rowcolor{bleudefrance}
        
            \multicolumn{3}{c|}{\color{aliceblue}\Large\textbf{Plataforma de Colaboración para Equipos de Software}}\\
            \hline
            \rowcolor{bleudefrance} \color{aliceblue}{ \textbf{Logo}} & \color{aliceblue}\textbf{CARACTERISTICA} & \color{aliceblue}\textbf{JUSTIFICACION} \\
            \hline
            \endfirsthead
            
            %Esto es para que la tabla se repita en cada pagina, cuando es muy larga
            \rowcolor{bleudefrance}
            \hline 
            \rowcolor{bleudefrance} \color{aliceblue}{ \textbf{Logo}} & \color{aliceblue}\textbf{CARACTERISTICA} & \color{aliceblue}\textbf{JUSTIFICACION} \\           
            \hline
            \endhead
    
% Fila de datos \includegraphics[width=\linewidth]{\logoqr}
    \raisebox{-\totalheight}{\includegraphics[width=3cm]{\github}} & 
    \begin{itemize}
        \item Revisión de Código: Colabora de forma continua, fusiona con confianza y ofrece
        código de calidad.
        \item Permisos de Rama: Proporciona un control de acceso granular para tu equipo y
        garantiza.

    \end{itemize} & 
    Github es el lugar perfecto para trabajar conjuntamente en el proyecto
    ,invitando a todo el equipo a colaborar.
    Donde cada integrante cuenta con su contribucion al desarrollo del software.\\
    \hline

            \hline
            \rowcolor{bleudefrance} \multicolumn{3}{c|}{} \\
            \hline
            
            \end{longtable}
%----------------------------------





        % Herramienta de Documentacion 
       
        \begin{longtable}{|p{3cm}|p{6cm}|p{6cm}|}
            %\caption{Definiciones iniciales - Sprint 0 (opcional)}\\
            \hline
            \rowcolor{bleudefrance}
        
            \multicolumn{3}{c|}{\color{aliceblue}\Large\textbf{Software de Documentación y Reporte:PROJECT LIBRE}}\\
            \hline
            \rowcolor{bleudefrance} \color{aliceblue}{ \textbf{Logo}} & \color{aliceblue}\textbf{CARACTERISTICA} & \color{aliceblue}\textbf{JUSTIFICACION} \\
            \hline
            \endfirsthead
            
            %Esto es para que la tabla se repita en cada pagina, cuando es muy larga
            \rowcolor{bleudefrance}
            \hline 
            \rowcolor{bleudefrance} \color{aliceblue}{ \textbf{Logo}} & \color{aliceblue}\textbf{CARACTERISTICA} & \color{aliceblue}\textbf{JUSTIFICACION} \\           
            \hline
            \endhead
    
% Fila de datos \includegraphics[width=\linewidth]{\logoqr}
    \raisebox{-\totalheight}{\includegraphics[width=3cm]{\project}} & 
    \begin{itemize}
        \item Interfaz de Usuario: Similar a Microsoft Project, facilitando la transición para
        usuarios familiares con esa herramienta.
        \item Diagrama de Gantt: Permite la creación y gestión de cronogramas de proyectos.
        \item Compatibilidad: Compatible con archivos de Microsoft Project (.mpp) y otros
        formatos de archivo comunes.
        \item Gratis y de Código Abierto: No tiene costo y permite modificaciones por parte de
        la comunidad de desarrolladores.

    \end{itemize} & 
    ProjectLibre es una excelente elección para la gestión de proyectos debido a su similitud
con Microsoft Project, lo que facilita la adopción por parte de usuarios con experiencia en ese
software, además de ser gratuito y de código abierto, lo que permite su personalización y
accesibilidad sin costo. Su capacidad para crear diagramas de Gantt, gestionar recursos y costos,
establecer líneas base y colaborar en equipo lo convierte en una herramienta robusta y versátil
para diversas necesidades de gestión de proyectos.\\
    \hline

            \hline
            \rowcolor{bleudefrance} \multicolumn{3}{c|}{} \\
            \hline
            
            \end{longtable}
%----------------------------------
    \textbf{e) Definir infraestructura tecnológica (hardware y software) para el proceso de desarrollo del software}
   
    




        % python
       
        \begin{longtable}{|p{3cm}|p{6cm}|p{6cm}|}
            %\caption{Definiciones iniciales - Sprint 0 (opcional)}\\
            \hline
            \rowcolor{bleudefrance}
        
            \multicolumn{3}{c|}{\color{aliceblue}\Large\textbf{Lenguaje de Programación: Python}}\\
            \hline
            \rowcolor{bleudefrance} \color{aliceblue}{ \textbf{Logo}} & \color{aliceblue}\textbf{CARACTERISTICA} & \color{aliceblue}\textbf{JUSTIFICACION} \\
            \hline
            \endfirsthead
            
            %Esto es para que la tabla se repita en cada pagina, cuando es muy larga
            \rowcolor{bleudefrance}
            \hline 
            \rowcolor{bleudefrance} \color{aliceblue}{ \textbf{Logo}} & \color{aliceblue}\textbf{CARACTERISTICA} & \color{aliceblue}\textbf{JUSTIFICACION} \\           
            \hline
            \endhead
    
% Fila de datos \includegraphics[width=\linewidth]{\logoqr}
    \raisebox{-\totalheight}{\includegraphics[width=3cm]{\python}} & 
    \begin{itemize}
        \item Los desarrolladores pueden leer y comprender fácilmente los programas de Python
        debido a su sintaxis básica similar a la del inglés.
        \item Python cuenta con una gran biblioteca estándar que contiene códigos reutilizables
        para casi cualquier tarea. 
        \item Hay muchos recursos útiles disponibles en Internet si desea aprender Python.
 

    \end{itemize} & 
    Lenguaje de programación versátil, fácil de aprender y con una amplia 
    comunidad de apoyo. Su sintaxis clara y legible, junto con su 
    capacidad para ser utilizado en diversas plataformas y sectores, 
    lo convierten en una opción popular para el desarrollo de aplicaciones,
    análisis de datos, inteligencia artificial y mucho más. \\
    \hline

            \hline
            \rowcolor{bleudefrance} \multicolumn{3}{c|}{} \\
            \hline
            
            \end{longtable}
%----------------------------------

        % javascript
       
        \begin{longtable}{|p{3cm}|p{6cm}|p{6cm}|}
            %\caption{Definiciones iniciales - Sprint 0 (opcional)}\\
            \hline
            \rowcolor{bleudefrance}
        
            \multicolumn{3}{c|}{\color{aliceblue}\Large\textbf{Lenguaje de Programación: JavaScript}}\\
            \hline
            \rowcolor{bleudefrance} \color{aliceblue}{ \textbf{Logo}} & \color{aliceblue}\textbf{CARACTERISTICA} & \color{aliceblue}\textbf{JUSTIFICACION} \\
            \hline
            \endfirsthead
            
            %Esto es para que la tabla se repita en cada pagina, cuando es muy larga
            \rowcolor{bleudefrance}
            \hline 
            \rowcolor{bleudefrance} \color{aliceblue}{ \textbf{Logo}} & \color{aliceblue}\textbf{CARACTERISTICA} & \color{aliceblue}\textbf{JUSTIFICACION} \\           
            \hline
            \endhead
    
% Fila de datos \includegraphics[width=\linewidth]{\logoqr}
    \raisebox{-\totalheight}{\includegraphics[width=3cm]{\javas}} & 
    \begin{itemize}
        \item Es Liviano (usa pocos recursos).
        \item Multiplataforma, ya que se puede utilizar en Windows, Linux o Mac o en el
        navegador de tu preferencia.
        \item Orientado a objetos y eventos.
 

    \end{itemize} & 
    JavaScript va generando tendencia en poder desarrollarse no solo en el navegador, sino
también en el servidor, pero para ello necesitaremos y entorno de ejecución. \\
    \hline

            \hline
            \rowcolor{bleudefrance} \multicolumn{3}{c|}{} \\
            \hline
            
            \end{longtable}




%----------------------------------

        % Flask
       
        \begin{longtable}{|p{3cm}|p{6cm}|p{6cm}|}
            %\caption{Definiciones iniciales - Sprint 0 (opcional)}\\
            \hline
            \rowcolor{bleudefrance}
        
            \multicolumn{3}{c|}{\color{aliceblue}\Large\textbf{Framework de python : FLASK}}\\
            \hline
            \rowcolor{bleudefrance} \color{aliceblue}{ \textbf{Logo}} & \color{aliceblue}\textbf{CARACTERISTICA} & \color{aliceblue}\textbf{JUSTIFICACION} \\
            \hline
            \endfirsthead
            
            %Esto es para que la tabla se repita en cada pagina, cuando es muy larga
            \rowcolor{bleudefrance}
            \hline 
            \rowcolor{bleudefrance} \color{aliceblue}{ \textbf{Logo}} & \color{aliceblue}\textbf{CARACTERISTICA} & \color{aliceblue}\textbf{JUSTIFICACION} \\           
            \hline
            \endhead
    
% Fila de datos \includegraphics[width=\linewidth]{\logoqr}
    \raisebox{-\totalheight}{\includegraphics[width=3cm]{\flask}} & 
    \begin{itemize}
        \item Peso ligero.
        \item Sistema de pruebas unitarias
        \item OSoporte de extensiones.
        \item Compatibilidad con WSGI.
 

    \end{itemize} & 
    Flask es un microframework que no requiere librerías externas para implementar sus
    funcionalidades.
    Flask viene con un montón de herramientas, tecnologías y bibliotecas necesarias para el
desarrollo de aplicaciones web. \\
    \hline

            \hline
            \rowcolor{bleudefrance} \multicolumn{3}{c|}{} \\
            \hline
            
            \end{longtable}



%----------------------------------

        % hmtl
       
        \begin{longtable}{|p{3cm}|p{6cm}|p{6cm}|}
            %\caption{Definiciones iniciales - Sprint 0 (opcional)}\\
            \hline
            \rowcolor{bleudefrance}
        
            \multicolumn{3}{c|}{\color{aliceblue}\Large\textbf{Lenguaje de Marcado : HTML 5}}\\
            \hline
            \rowcolor{bleudefrance} \color{aliceblue}{ \textbf{Logo}} & \color{aliceblue}\textbf{CARACTERISTICA} & \color{aliceblue}\textbf{JUSTIFICACION} \\
            \hline
            \endfirsthead
            
            %Esto es para que la tabla se repita en cada pagina, cuando es muy larga
            \rowcolor{bleudefrance}
            \hline 
            \rowcolor{bleudefrance} \color{aliceblue}{ \textbf{Logo}} & \color{aliceblue}\textbf{CARACTERISTICA} & \color{aliceblue}\textbf{JUSTIFICACION} \\           
            \hline
            \endhead
    
% Fila de datos \includegraphics[width=\linewidth]{\logoqr}
    \raisebox{-\totalheight}{\includegraphics[width=3cm]{\html}} & 
    \begin{itemize}
        \item APIs de Web Storage (localStorage y sessionStorage) para almacenar datos en
        el navegador.
        \item API para obtener la ubicación del usuario.
        \item Soporte Offline: Application Cache y Service Workers para aplicaciones web
        offline.

 

    \end{itemize} & 
    HTML5 es esencial en el desarrollo web moderno por sus nuevas etiquetas semánticas,
soporte nativo para audio y video, capacidades gráficas avanzadas, mejoras en formularios y
almacenamiento local, y funcionalidades offline. Estas características, junto con su
compatibilidad con navegadores modernos y mejor rendimiento, eficientes y accesibles. \\
    \hline

            \hline
            \rowcolor{bleudefrance} \multicolumn{3}{c|}{} \\
            \hline
            
            \end{longtable}
%----------------------------------

        % CSS
       
        \begin{longtable}{|p{3cm}|p{6cm}|p{6cm}|}
            %\caption{Definiciones iniciales - Sprint 0 (opcional)}\\
            \hline
            \rowcolor{bleudefrance}
        
            \multicolumn{3}{c|}{\color{aliceblue}\Large\textbf{Lenguaje de Diseño : CSS}}\\
            \hline
            \rowcolor{bleudefrance} \color{aliceblue}{ \textbf{Logo}} & \color{aliceblue}\textbf{CARACTERISTICA} & \color{aliceblue}\textbf{JUSTIFICACION} \\
            \hline
            \endfirsthead
            
            %Esto es para que la tabla se repita en cada pagina, cuando es muy larga
            \rowcolor{bleudefrance}
            \hline 
            \rowcolor{bleudefrance} \color{aliceblue}{ \textbf{Logo}} & \color{aliceblue}\textbf{CARACTERISTICA} & \color{aliceblue}\textbf{JUSTIFICACION} \\           
            \hline
            \endhead
    
% Fila de datos \includegraphics[width=\linewidth]{\logoqr}
    \raisebox{-\totalheight}{\includegraphics[width=3cm]{\css}} & 
    \begin{itemize}
        \item Permite definir colores, fuentes, espaciado, y diseño visual de páginas web.
        \item Ayuda a mantener el HTML limpio al separar el diseño visual en un archivo CSS.
        \item Facilita la creación de diseños que se adaptan automáticamente a diferentes tamaños de
        pantalla.

 

    \end{itemize} & 
    CSS es fundamental en el desarrollo web porque permite separar la estructura y el
contenido de una página HTML de su presentación visual. Esto no solo mejora la mantenibilidad
del código al facilitar cambios de diseño sin alterar el contenido, sino que también proporciona
control preciso sobre aspectos visuales como colores, fuentes, espaciado y disposición de
elementos. \\
    \hline

            \hline
            \rowcolor{bleudefrance} \multicolumn{3}{c|}{} \\
            \hline
            
            \end{longtable}       

%----------------------------------

        % framerork CSS tailwind
       
        \begin{longtable}{|p{3cm}|p{6cm}|p{6cm}|}
            %\caption{Definiciones iniciales - Sprint 0 (opcional)}\\
            \hline
            \rowcolor{bleudefrance}
        
            \multicolumn{3}{c|}{\color{aliceblue}\Large\textbf{Framework de Diseño : TAILWIND }}\\
            \hline
            \rowcolor{bleudefrance} \color{aliceblue}{ \textbf{Logo}} & \color{aliceblue}\textbf{CARACTERISTICA} & \color{aliceblue}\textbf{JUSTIFICACION} \\
            \hline
            \endfirsthead
            
            %Esto es para que la tabla se repita en cada pagina, cuando es muy larga
            \rowcolor{bleudefrance}
            \hline 
            \rowcolor{bleudefrance} \color{aliceblue}{ \textbf{Logo}} & \color{aliceblue}\textbf{CARACTERISTICA} & \color{aliceblue}\textbf{JUSTIFICACION} \\           
            \hline
            \endhead
    
% Fila de datos \includegraphics[width=\linewidth]{\logoqr}
    \raisebox{-\totalheight}{\includegraphics[width=3cm]{\tailwind}} & 
    \begin{itemize}
        \item Utiliza clases de CSS predefinidas para aplicar estilos directamente en el HTML.
        \item Permite construir diseños complejos y responsivos sin escribir CSS personalizado.
        \item Es altamente personalizable a través de un archivo de configuración para adaptarse a


 

    \end{itemize} & 
    Tailwind CSS ofrece una metodología eficiente y directa para el diseño de interfaces web
al proporcionar un conjunto extenso de clases de utilidad predefinidas. Esta aproximación
permite a los desarrolladores crear y personalizar rápidamente estilos visuales consistentes y
responsivos sin la necesidad de escribir CSS personalizado, optimizando así el flujo de trabajo de
desarrollo y facilitando la mantenibilidad del código. \\
    \hline

            \hline
            \rowcolor{bleudefrance} \multicolumn{3}{c|}{} \\
            \hline
            
            \end{longtable}       

%----------------------------------

        % SGBD Postgresql
       
        \begin{longtable}{|p{3cm}|p{6cm}|p{6cm}|}
            %\caption{Definiciones iniciales - Sprint 0 (opcional)}\\
            \hline
            \rowcolor{bleudefrance}
        
            \multicolumn{3}{c|}{\color{aliceblue}\Large\textbf{Gestor de Base de Datos : POSTGRESQL }}\\
            \hline
            \rowcolor{bleudefrance} \color{aliceblue}{ \textbf{Logo}} & \color{aliceblue}\textbf{CARACTERISTICA} & \color{aliceblue}\textbf{JUSTIFICACION} \\
            \hline
            \endfirsthead
            
            %Esto es para que la tabla se repita en cada pagina, cuando es muy larga
            \rowcolor{bleudefrance}
            \hline 
            \rowcolor{bleudefrance} \color{aliceblue}{ \textbf{Logo}} & \color{aliceblue}\textbf{CARACTERISTICA} & \color{aliceblue}\textbf{JUSTIFICACION} \\           
            \hline
            \endhead
    
% Fila de datos \includegraphics[width=\linewidth]{\logoqr}
    \raisebox{-\totalheight}{\includegraphics[width=3cm]{\postgre}} & 
    \begin{itemize}
        \item Es una base de datos 100 ACID.
        \item Incluye herencia entre tablas, por lo que a este gestor de bases de datos se le incluye
        entre los gestores objeto-relacionales.  
        \item Copias de seguridad en caliente (Online/hot backups)


 

    \end{itemize} & 
    Se eligió PostgreSQL porque es más fácil al trabajar en la nube, es open source al igual
que uno de los objetivos de este proyecto. También el equipo de desarrollo tiene experiencia en
este motor de BD. \\
    \hline

            \hline
            \rowcolor{bleudefrance} \multicolumn{3}{c|}{} \\
            \hline
            
            \end{longtable}    



%----------------------------------

        % AWS
       
        \begin{longtable}{|p{3cm}|p{6cm}|p{6cm}|}
            %\caption{Definiciones iniciales - Sprint 0 (opcional)}\\
            \hline
            \rowcolor{bleudefrance}
        
            \multicolumn{3}{c|}{\color{aliceblue}\Large\textbf{Despliegue del proyecto : AWS (AMAZON WEB SERVICES)}}\\
            \hline
            \rowcolor{bleudefrance} \color{aliceblue}{ \textbf{Logo}} & \color{aliceblue}\textbf{CARACTERISTICA} & \color{aliceblue}\textbf{JUSTIFICACION} \\
            \hline
            \endfirsthead
            
            %Esto es para que la tabla se repita en cada pagina, cuando es muy larga
            \rowcolor{bleudefrance}
            \hline 
            \rowcolor{bleudefrance} \color{aliceblue}{ \textbf{Logo}} & \color{aliceblue}\textbf{CARACTERISTICA} & \color{aliceblue}\textbf{JUSTIFICACION} \\           
            \hline
            \endhead
    
% Fila de datos \includegraphics[width=\linewidth]{\logoqr}
    \raisebox{-\totalheight}{\includegraphics[width=3cm]{\aws}} & 
    \begin{itemize}
        \item Varias ubicaciones, Amazon EC2 ofrece la posibilidad de colocar instancias en distintas
        ubicaciones.    
        \item IHora de alta precisión con el servicio de Amazon Time Sync, ofrece una fuente de hora
        de alta precisión, fiabilidad y disponibilidad para los servicios de AWS, incluidas las
        instancias EC2.  



 

    \end{itemize} & 
    Amazon Elastic Compute Cloud (Amazon EC2) proporciona capacidad de computación
    escalable bajo demanda en la nube de Amazon Web Services (AWS). El uso de Amazon EC2
    reduce los costos de hardware para que pueda desarrollar e implementar aplicaciones con mayor
    rapidez. \\
    \hline

            \hline
            \rowcolor{bleudefrance} \multicolumn{3}{c|}{} \\
            \hline
            
            \end{longtable}    


%----------------------------------

        % Traslate
       
        \begin{longtable}{|p{3cm}|p{6cm}|p{6cm}|}
            %\caption{Definiciones iniciales - Sprint 0 (opcional)}\\
            \hline
            \rowcolor{bleudefrance}
        
            \multicolumn{3}{c|}{\color{aliceblue}\Large\textbf{Inteligencia Artificial  : GOOGLE TRASLATE}}\\
            \hline
            \rowcolor{bleudefrance} \color{aliceblue}{ \textbf{Logo}} & \color{aliceblue}\textbf{CARACTERISTICA} & \color{aliceblue}\textbf{JUSTIFICACION} \\
            \hline
            \endfirsthead
            
            %Esto es para que la tabla se repita en cada pagina, cuando es muy larga
            \rowcolor{bleudefrance}
            \hline 
            \rowcolor{bleudefrance} \color{aliceblue}{ \textbf{Logo}} & \color{aliceblue}\textbf{CARACTERISTICA} & \color{aliceblue}\textbf{JUSTIFICACION} \\           
            \hline
            \endhead
    
% Fila de datos \includegraphics[width=\linewidth]{\logoqr}
    \raisebox{-\totalheight}{\includegraphics[width=3cm]{\traslate}} & 
    \begin{itemize}
        \item Traducción de texto: Puedes ingresar texto en un idioma y obtener la traducción
        correspondiente en otro idioma. La aplicación admite una amplia variedad de idiomas.    
        \item Traducción por voz: Google Translate permite la traducción de voz. Puedes hablar en
        un idioma y la aplicación traducirá tu discurso al idioma deseado.
        \item Traducción de conversaciones en tiempo real: La función de "Conversación" permite
        la traducción de voz en tiempo real durante una conversación. Esto puede ser útil para
        comunicarse con personas que hablan un idioma diferente.




 

    \end{itemize} & 
    Google Traslate es útil para traducciones rápidas y sencillas, especialmente cuando necesitas entender 
    el significado general de un texto o una conversación. También es una 
    herramienta gratuita y accesible que puede ayudarte a comunicarte con personas que hablan otros idiomas, 
    incluso sin conocer el idioma.\\
    \hline

            \hline
            \rowcolor{bleudefrance} \multicolumn{3}{c|}{} \\
            \hline
            
            \end{longtable}               

%------------------------------------------------------------------
\newpage

    \item \large\textbf{Definir patrón de desarrollo que definan actividades técnicas de desarrollo que serán realizadas
    durante el sprint}
    
    \begin{enumerate}
        
        \item \textbf {Arquitectura Base}\par
        Se adoptará el patrón de arquitectura \textbf{MVC}. Cada sprint técnico incluirá:
        \begin{itemize}
            \item \textbf{Modelo}: Definición de la estructura de datos y la lógica de negocio.
            \item \textbf{Vista}: Diseño de la interfaz de usuario y la presentación de datos.
            \item \textbf{Controlador}: Implementación de la lógica de control y la interacción entre el modelo y la vista.
        \end{itemize}

        \item \textbf{Gestión de la Configuración}\par
        Se utilizarán las siguientes herramientas y procesos para la gestión de la configuración:

        \begin{itemize}
            \item \textbf{Control de Versiones}: Utilizar Git para el control de versiones del código fuente.
            \item \textbf{Gestión de Requisitos}: Utilizar Jira para la gestión de requisitos y tareas.
            \item \textbf{Documentación}: Utilizar Confluence para la documentación del proyecto.
            \item \textbf{Gestión de Cambios}: Establecer un proceso de gestión de cambios para controlar las modificaciones en el código y los requisitos.
            \item \textbf{Gestión de Incidencias}: Utilizar Jira para la gestión de incidencias y errores.
            \item \textbf{Gestión de Pruebas}: Utilizar herramientas de pruebas automatizadas para garantizar la calidad del software.
            \item \textbf{Gestión de Despliegue}: Utilizar AWS para el despliegue de la aplicación.
        \end{itemize}
        \item \textbf{Instalación de Herramientas}\par
        \begin{itemize}
            \item \textbf{Instalación de Python}: Instalar la última versión de Python en el entorno de desarrollo.
            \item \textbf{Instalación de Flask}: Instalar Flask y sus dependencias utilizando pip.
            \item \textbf{Instalación de PostgreSQL}: Instalar PostgreSQL y configurar la base de datos.
            \item \textbf{Instalación de Tailwind CSS}: Instalar Tailwind CSS y configurarlo en el proyecto.
            \item \textbf{Instalación de Google Translate API}: Configurar la API de Google Translate para la traducción automática.
            \item \textbf{Instalación de GitHub}: Configurar GitHub para el control de versiones y la colaboración en el proyecto.
            \item \textbf{Instalación de Enterprise Architect}: Instalar Enterprise Architect para el modelado UML.
            \item \textbf{Instalación de ProjectLibre}: Instalar ProjectLibre para la gestión de proyectos.
            \item \textbf{Instalación de Postman}: Instalar Postman para probar
            \item \textbf{Instalación de Docker}: Instalar Docker para la creación y gestión de contenedores.
            \item \textbf{Instalación de Visual Studio Code}: Instalar Visual Studio Code como editor de código.
            \item \textbf{Creación de proyecto en Jira}: Crear un proyecto en Jira para la gestión de tareas y seguimiento del progreso.
            \item \textbf{Crearción de Instancia en AWS}: Crear una instancia EC2 en AWS para el despliegue de la aplicación.
        \end{itemize}
        \item \textbf{Configuración de Entorno de Desarrollo}\par
        \begin{itemize}
            \item \textbf{Configuración de Entorno Virtual}: Crear un entorno virtual para el proyecto utilizando venv o virtualenv.
            \item \textbf{Configuración de Dependencias}: Instalar las dependencias del proyecto utilizando pip y un archivo requirements.txt.
            \item \textbf{Configuración de Base de Datos}: Configurar la conexión a la base de datos PostgreSQL en el archivo de configuración del proyecto.
            \item \textbf{Configuración de Variables de Entorno}: Configurar las variables de entorno necesarias para el proyecto, como claves API y credenciales de base de datos.
            \item \textbf{Configuración de Docker}: Crear un Dockerfile y un archivo docker-compose.yml para la creación y gestión de contenedores.
            \item \textbf{Configuración de Git}: Configurar el repositorio de Git y crear ramas para el desarrollo.
            \item \textbf{Configuración de Postman}: Configurar Postman para probar
            \item \textbf{Configuración de Google Translate API}: Configurar la API de Google Translate para la traducción automática.
            \item \textbf{Configuración de Enterprise Architect}: Configurar Enterprise Architect para el modelado UML.
            \item \textbf{Configuración de Jira}: Configurar Jira para la gestión de tareas y seguimiento del progreso.
            \item \textbf{Configuración de GitHub}: Configurar GitHub para el control de versiones y la colaboración en el proyecto.
            \item \textbf{Configuración de Instancia en AWS}: Configurar la instancia EC2 en AWS para el despliegue de la aplicación.
        \end{itemize}

    \end{enumerate}

%------------------------------------------------------------------
\newpage
    Realizar en grupo
    \item \large\textbf{Esbozar modelos inicial de: Contexto, de Datos y Arquitectura}
        \begin{itemize}
            \item \textbf{Modelo de Contexto}: Definir el contexto del sistema, incluyendo los actores externos y las interacciones con el sistema.
            \item \textbf{Modelo de Datos}: Definir el modelo de datos del sistema, incluyendo las entidades, atributos y relaciones entre ellas.
            \item \textbf{Modelo de Arquitectura}: Definir la arquitectura del sistema, incluyendo los componentes, módulos y su interacción.
        \end{itemize}

    \item \large\textbf{Definición criterios de calidad}
    \begin{itemize}
        
        \item \textbf{Correcto:} Cumple con las necesidades que requiere el cliente\par
        \item \textbf{Eficiente:} El usuario no necesita esforzarse por usar el software
        \item \textbf{Fiabilidad:} El software cumple con las funciones que se establecieron.
        \item \textbf{Usabilidad:} El software es fácil de usar y entender.
        \item \textbf{Manteniebilidad:} El software es fácil de mantener y actualizar.
        \item \textbf{Segurida e integridad:} El software es seguro y protege la información del usuario.
        \item \textbf{Portabilidad:} El software puede ser utilizado en diferentes plataformas.

    \end{itemize}

%----------------------------------

\newpage

\item \large\textbf{Generar el Product Backlog (artefacto) , ver ejemplo de formato F3}

\newcolumntype{P}[1]{>{\centering\arraybackslash}p{#1}}
% Tabla 1
\begin{longtable}{|p{4cm}|P{5cm}|P{5.3cm}|P{2.2cm}|}
    \hline
    \rowcolor{bleudefrance}
    \multicolumn{4}{|c|}{\color{aliceblue}\Large\textbf{Product Backlog}}\\
    \hline
    \rowcolor{lightgray} \textbf{Proyecto} & \multicolumn{3}{l|}{ Sistema Avanzado de Transcripción y Traducción Multilingue 
    con IA} \\
    \hline
    \rowcolor{aliceblue} \textbf{Product Owner} & \multicolumn{3}{l|}{GUZMAN HONOR WILSON } \\
    \hline
    \rowcolor{lightgray} \textbf{Versión} & \textbf{ 1.0} & \textbf{Fecha} & \textbf{11/05/2025} \\
    \hline

\end{longtable} 
\vspace{-2em} % Elimina espacio entre tablas para que paresca solo una tabla

% Tabla 2
% para definir el ancho de las columnas
%\newcolumntype{P}[1]{>{\centering\arraybackslash}p{#1}}
% Tabla para los ítems del backlog con anchos personalizados
\begin{longtable}{|P{0.3cm}|P{5cm}|P{9cm}|P{2.2cm}|}
    \hline
    \rowcolor{bleudefrance}
    \color{aliceblue}\textbf{id} & 
    \color{aliceblue}\textbf{Nombre corto del requerimiento} & 
    \color{aliceblue}\textbf{Descripción del requerimiento funcional
     usando \textless como\textgreater{} … 
     \textless quiero\textgreater{} … \textless para\textgreater{}} & 
     \color{aliceblue}\textbf{Prioridad} \\
    \hline
    \endfirsthead
    \rowcolor{bleudefrance}
    \color{aliceblue}\textbf{id} & 
    \color{aliceblue}\textbf{Nombre corto del requerimiento} & 
    \color{aliceblue}\textbf{Descripción del requerimiento funcional
     usando \textless como\textgreater{} … 
     \textless quiero\textgreater{} … \textless para\textgreater{}} & 
     \color{aliceblue}\textbf{Prioridad} \\
    \hline
    \endhead

    % Aquí podés poner tus filas de datos
    \hline
    1 
    & Registrar usuario 
    & Como usuario, quiero registrarme en el sistema para 
    poder acceder a todas las funcionalidades. 
    & Baja \\
    \hline
    \rowcolor{lightblue} 2 
    & Iniciar sesión 
    & Como usuario, quiero poder iniciar sesión 
    en el sistema para acceder a las funcionalidades. 
    & Baja \\
    \hline
    3 
    & Interfaz de usuario 
    & Como usuario, quiero una interfaz amigable y
    fácil de usar para poder interactuar con el
    sistema sin complicaciones. 
    & Baja \\
    \hline
    \rowcolor{lightblue} 4 
    & Inabilitar cuenta 
    & Como administrador, quiero inhabilitar cuentas de usuario 
    para evitar el uso indebido del sistema. 
    & Baja \\
    \hline
    5 
    & Cargar audio 
    & Como usuario, quiero cargar un archivo de audio para que 
    el sistema lo transcriba y traduzca. 
    & Alta \\
    \hline
    \rowcolor{lightblue} 6 
    & Transcripción de audio 
    & Como usuario, quiero ver la transcripción del audio
    en pantalla para tener en formato texto 
    & Alta \\
    \hline
    7 
    & Traducir audio 
    & Como usuario, quiero ver la traducción del audio en pantalla en formato texto 
    para poder comprender el contenido. 
    & Alta \\
    \hline
    8 
    & Generar y exportar audio traducido 
    & Como usuario, quiero poder escuchar y descagar el audio traducido 
    para volver a usarlo en otra ocasion. 
    & Alta \\
    \hline
    \rowcolor{lightblue} 9 
    & Crear Plan 
    & Como administrador, quiero poder crear un plan de suscripción 
    para los usuarios para que puedan acceder a diferentes 
    funcionalidades según el plan elegido. 
    & Media \\
    \hline
    10 
    & Modificar Plan 
    & Como administrador, quiero poder modificar un plan de suscripción 
    para actualizar los precios o funcionalidades. 
    & Media \\
    \hline
    \rowcolor{lightblue} 11 
    & Eliminar Plan 
    & Como administrador, quiero poder eliminar un plan de suscripción
    para evitar que los usuarios se suscriban a él. 
    & Media \\
    \hline
    12 
    & Capturar audio 
    & Como usuario, quiero poder capturar audio en tiempo real
    para que el sistema lo transcriba y traduzca. 
    & Alta \\
    \hline
    \rowcolor{lightblue} 13 
    & Transcripción en tiempo real 
    & Como usuario, quiero ver la transcripción del audio del presentador 
    en pantalla para poder seguir el contenido aunque no escuche claramente. 
    & Alta \\
    \hline
    14 
    & Traducción en tiempo real 
    & Como usuario, quiero ver la traducción del audio del presentador 
    en pantalla para poder seguir el contenido aunque no escuche claramente. 
    & Alta \\
    \hline

    \rowcolor{lightblue}15 
    & Registrar suscripción 
    & Como usuario, quiero poder registrarme en un plan de suscripción 
    para acceder a todas las funcionalidades del sistema. 
    & Alta \\
    \hline
    16 
    & Modificar suscripción 
    & Como usuario, quiero poder modificar mi plan de suscripción 
    para cambiar a otro plan. 
    & Alta \\
    \hline
    \rowcolor{lightblue}17 
    & Cancelar suscripción 
    & Como usuario, quiero poder cancelar mi plan de suscripción 
    para dejar de pagar por el servicio. 
    & Alta \\
    \hline
    18 
    & Gestionar pagos 
    & Como usuario, quiero poder gestionar mis pagos 
    para poder ver mis facturas y pagos realizados. 
    & Media \\
    \hline
    \rowcolor{lightblue} 19 
    & Cargar video 
    & Como usuario, quiero poder cargar un archivo de video 
    para que el sistema lo transcriba y traduzca. 
    & Media \\
    \hline
    20
    & Transcribir video 
    & Como Usuario, quiero ver la transcripción del video en pantalla 
    para que pueda ser traducido posteriormente. 
    & Media \\
    \hline
    \rowcolor{lightblue} 21
    & Traducir video 
    & Como Usuario, quiero ver la traducción del video cargado en pantalla 
    para poder comprender el contenido. 
    & Media \\
    \hline
    22
    & Generar y exportar audio traducido 
    & Como usuario, quiero poder escuchar y descagar el audio traducido 
    para volver a usarlo en otra ocasion. 
    & Media \\ 
    \hline
        \hline
        \rowcolor{bleudefrance} \multicolumn{4}{c}{} \\
        \hline  
    \end{longtable}  






  \end{enumerate}

\end{doublespace}