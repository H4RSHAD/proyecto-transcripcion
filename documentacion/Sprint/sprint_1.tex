
% Mostrar hasta paragraph (cuarto nivel) en la tabla de contenido
\setcounter{tocdepth}{5}
\setcounter{secnumdepth}{5}

% Personalizar paragraph para que actúe como una mini-sección
\titleformat{\paragraph}
  {\normalfont\normalsize\bfseries} % estilo
  {\theparagraph}                   % numeración
  {1em}                             % espacio entre número y título
  {}                                % código antes del título

% Asegura que los paragraph tengan su propia línea
\titlespacing*{\paragraph}{0pt}{1.5ex plus 0.5ex minus .2ex}{1em}

%----------------------------------------------------------------------------
\begin{doublespace} 

    % Definición de colores personalizados
    \definecolor{lightgray}{rgb}{0.9, 0.9, 0.9}
    \definecolor{lightblue}{rgb}{0.8, 0.9, 1.0}
    \definecolor{aliceblue}{rgb}{0.94, 0.97, 1.0}
    \definecolor{bleudefrance}{rgb}{0.19, 0.55, 0.91}

    \renewcommand{\arraystretch}{1.6} % Espaciado entre filas


\subsection{SPRINT N° 1}
\subsubsection{TAREAS A REALIZAR DURANTE EL SPRINT PLANNING (R-2)}




%DE LA DOCUMENTACION DE SCRUM DEL ING. MARTINEZ
\newcolumntype{P}[1]{>{\centering\arraybackslash}p{#1}}
% Tabla 1
\renewcommand{\arraystretch}{1.1} % Espaciado entre filas
%\setlength{\tabcolsep}{5pt} % Espaciado entre columnas








\begin{longtable}{|c|p{15cm}|}
%\caption{Definiciones iniciales - Sprint 0 (opcional)}\\
\hline
\rowcolor{bleudefrance}

\multicolumn{2}{c|}{\color{aliceblue}\Large\textbf{Tareas a realizar durante el Sprint Planning}}\\
\hline
\rowcolor{bleudefrance} \color{aliceblue}{\Large\textbf{Id}} & \Large\textbf{\color{aliceblue} Tarea}\\
\hline
\endfirsthead

%Esto es para que la tabla se repita en cada pagina, cuando es muy larga
\rowcolor{bleudefrance}
\hline 
\multicolumn{2}{c|}{\color{aliceblue}\textbf{Tareas a realizar durante el Sprint Planning} (continuación)}\\
\hline
\rowcolor{bleudefrance} \color{aliceblue}{\Large\textbf{Id}} & \color{aliceblue}\textbf{Tarea}\\
\hline
\endhead

a & El Product Owner (PO) propone cómo el producto podría aumentar su valor y utilidad en
el Sprint actual explicando las Historias de Usuario (HU) candidatas a desarrollar durante
la ejecución del Sprint (usar las 3C Card, Conversation, Confirmation)\par 
\color{blue}{Comentario}: \color{bleudefrance}{Aquí acontece la actividad de análisis esencial para comprender los requerimientos,
por lo que si se ve conveniente y útil generar/actualizar el modelo de contexto (puede ser el diagrama
general de casos de uso) para tener una vista más abstracta de lo que proporciona las Historias de
Usuario}\par
\color{black}{Ver ejemplo de formato para especificar historias de usuario F4}\\
\rowcolor{lightblue} b & El equipo Scrum define objetivo del Sprint\\
\hline
c & Los desarrolladores estiman los Puntos de Historia de Usuario para cada HU para saber
con cuanto se pueden comprometer a completar en el Sprint (usar Planning Poker) \\
\hline
\rowcolor{lightblue} d & Los desarrolladores seleccionan las HU a desarrollar en el presente Sprint\\
\hline
e & Los desarrolladores definen la solución en términos de diseño esencial
    \begin{enumerate}[label=\alph*)]
        \item Definir/Actualizar el diseño de la arquitectura de software\par
                Modelos resultados sugeridos:
                \begin{itemize}
                    \item Opción 1: Modelos en C4: Niveles 1, 2 y 3
                    \item Opción 2: Modelos en UML: Diagrama de paquetes y Diagrama de despliegue
                \end{itemize}
        \item Definir/Actualizar el diseño de datos\par
                Modelo resultado:
                \begin{itemize}
                    \item Modelo conceptual: Diagrama de clases en UML
                \end{itemize}

    \end{enumerate} 
    \color{blue}{Comentario}: \color{bleudefrance}{Aquí acontece actividades diseño esencial para definir la solución y como será
    implementado el software, también se puede detallar y/o actualizar el diseño durante la ejecución
    del Sprint (R-3)}\\

\hline
\rowcolor{lightblue} f & Generar el Sprint Backlog (artefacto), Ver formato ejemplo F5 \\
\hline
\rowcolor{bleudefrance} \multicolumn{2}{c|}{} \\
\hline

\end{longtable}

\clearpage  %nueva pagina
%TAREAS A REALIZAR DURANTE EL SPRINT PLANNING (R-2)- NUESTRO SCRUM

% Usamos \addcontentsline para incluir el paragraph en el índice
\paragraph{\Large\textbf {Historia de Usuario (propone el Product Owner)}}
\addcontentsline{toc}{paragraph}{Historia de Usuario}
\subfile{1/R-2/a/Historia_de_usuario}

\clearpage  %nueva pagina
% Usamos \addcontentsline para incluir el paragraph en el índice
\paragraph{\Large\textbf {Objetivo del Sprint (definido por el Scrum Máster)}}
\addcontentsline{toc}{paragraph}{Objetivo del Sprint}
\subfile{1/R-2/b/Objetivo_sprint_1}

%nueva pagina
\paragraph{\Large\textbf {Planning Poker (estimación de Puntos)}}
\addcontentsline{toc}{paragraph}{Planning Poker}
\subfile{1/R-2/c/Planning_poker}

\clearpage  %nueva pagina
\paragraph{\large\textbf {SELECCIÓN DE HISTORIAS DE USUARIOS (desarrolladores)}}
\addcontentsline{toc}{paragraph}{selecciona de Historias de Usuario}
\subfile{1/R-2/d/HU_a_desarrollar}

\clearpage  %nueva pagina
\paragraph{\Large\textbf {DISEÑO DE LA SOLUCIÓN (desarrolladores)}}
\addcontentsline{toc}{paragraph}{Diseño de la Solución}
\subfile{1/R-2/e/Diseño_solucion}





%---------------------------------------------------------------------------

\clearpage  %nueva pagina
%TAREAS A REALIZAR DURANTE EL SPRINT PLANNING (R-3)- NUESTRO SCRUM

\subsubsection{ACTIVIDADES A REALIZAR DURANTE LA EJECUCION DEL SPRINT (R-3)}
El propósito de la ejecución del Sprint es que el equipo de desarrollo trabaje en las tareas del
Sprint Backlog para cumplir con el objetivo del Sprint y entregar un incremento de software
potencialmente entregable. Durante esta fase, el equipo se enfoca en consolidar la solución,
implementar, probar y refinar las funcionalidades comprometidas, asegurando que se cumpla
con los criterios de calidad y esté listo para ser revisado en el Sprint Review.

\begin{enumerate}[label=\alph*)]
    \item \textbf{Cada desarrollador selecciona desde el Sprint Backlog que tareas realizará}\par
    Ejemplo de tareas para el desarrollo en la iteración del Sprint para generar un incremento.
    \item \textbf{Generación del incremento y despliegue}\par
    A la finalización de la ejecución del sprint se debe generar el incremento (artefacto)
    correspondiente al sprint, integrando el incremento resultante a la anterior para asegurarse
    que funcionan correctamente juntos, tomar en cuenta consideraciones siguientes:
    \begin{itemize}
        \item El incremento debe cumplir con el criterio de “Definition of Done” (DoD)
        \item Asegurarse de que todas las ramas de código estén integradas en la rama principal (main
        o master)
        \item Asegurarse de que el código siga las mejores prácticas de estilo y estándares del equipo.
        \item Preparar/monitorear el entorno de producción
        \item Considerar usar Contenedores (docker), automatizar el despliegue, uso de pipelines
        CI/CD (integración continua/despliegue continuo). Ejem. GitHub Actions, Jenkins, GitLab.
    \end{itemize}
\end{enumerate}
%---------------------------------------------------------------------------
\subsubsection{SCRUM DIARIO (DAILY SCRUM)}




%---------------------------------------------------------------------------
\subsubsection{REVISION DE SPRINT (R-4)}




%---------------------------------------------------------------------------
\subsubsection{RETROSPECTIVA DE SPRINT (R-5)}


\end{doublespace}