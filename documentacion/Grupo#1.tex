% En documento-principal.tex
\documentclass[a4paper]{article} % Formato de plantilla a utilizar
\usepackage[utf8]{inputenc}
\usepackage{subfiles} %para agregar otros documentos 
\usepackage[spanish]{babel}
\usepackage{setspace}               % Paquete para cambiar el interlineado
\usepackage[margin=2cm,top=2cm,includefoot]{geometry} % Margenes del documento
\usepackage{graphicx}               % Para insertar una imagen
\usepackage[table,xcdraw]{xcolor}   % Para detectar los colores
\usepackage[most]{tcolorbox}        % Para inserción de cuadros en portada y otros
\usepackage{fancyhdr}               % Definir el estilo de la pagina
\usepackage[hidelinks]{hyperref}    % Gestión de hipervinculos
%-------------------------------
\usepackage{enumitem} %para numerar con letras del abecedario
\usepackage{longtable}              % Para tablas largas
\usepackage{mdframed}            % Para crear cuadros de texto
\usepackage{url}
\usepackage{hyperref}               %si quieres enlaces clicables.
\usepackage{array}  % Para definir columnas con formato especial

\usepackage{multirow}
%----------------------------Horizontales----------------
\usepackage{fancyhdr}     % para encabezados y pies de página
\usepackage{pdflscape}    % para rotar horizontalmente
\usepackage{lscape}       % Para rotar tablas o páginas Horizontales

\usepackage{titlesec}     % Personalizar secciones
\usepackage{tocloft}      % Mejorar el índice (opcional)

% Declarar Colores
\definecolor{titulocolor}{HTML}{ab1e00}
%colores https://latexcolor.com/

% Declaración de variables globales
\newcommand{\logoportada}{img/Escudo_FICCT.png}
\newcommand{\logoqr}{img/qr.png}

% Adicional
%----------------ENCABEZADO--------------------
\setlength{\headheight}{40.2pt}
\pagestyle{fancy}
\fancyhf{}
\lhead{\textbf{INGENIERIA DE SOFTWARE I - SC}}\rhead{\textbf{Proyecto Final - Grupo N°1}}
\renewcommand{\headrulewidth}{2pt}
\fancyfoot[R]{\textbf{Página \thepage }} %  |\pageref{LastPage}} 
%-------------Fin ENCABEZADO ---------------------


% Comienzo del documento
\begin{document}
    %\cfoot{Página \thepage\|\thepage} % para que se agrege los numeros de paginas

    % -----------------------------CARATULA-----------------------------------------
    \begin{titlepage}
        \centering
        \Large\textbf{UNIVERSIDAD AUTONOMA 
        GABRIEL RENE MORENO}\par\vspace{1.5cm}

        \centering
        \includegraphics[width=0.4\textwidth]{\logoportada}\par\vspace{0.3cm}
        
        {\huge\bfseries\textcolor{titulocolor}{Grupo N°1}}\par\vspace{0.1cm}
        
        \begin{tcolorbox}[colback=red!5!white,colframe=red!75!black]
            \centering
            \Large \textbf{Sistema Avanzado de Transcripción y Traducción}\par
            \Large \textbf{Multilingue para Eventos Presenciales y Multimedia con Inteligencia Artificial }\par
            
        \end{tcolorbox}\par\vspace{1cm}

        % Definición de colores personalizados
        \definecolor{lightgray}{rgb}{0.9, 0.9, 0.9}
        \definecolor{lightblue}{rgb}{0.8, 0.9, 1.0}
        \definecolor{aliceblue}{rgb}{0.94, 0.97, 1.0}
        \definecolor{bleudefrance}{rgb}{0.19, 0.55, 0.91}
 
        \hfill  %debes de ajustar el espacio entre la tabla y el QR, no separarlo
        \begin{minipage}{0.79\textwidth} % 60% del ancho para la información
                    % Aquí va la tabla de los integrantes
            \begin{flushleft}
                
                \begin{longtable}[c]{|l|c|}
                    \hline
                    \rowcolor{bleudefrance} \color{aliceblue}\textbf{Nombre} & \color{aliceblue}\textbf{Registro} \\
                    \hline
                    \endfirsthead
                    
                    
                    \hline
                                            {GUZMAN HONOR WILSON} &             {214143325} \\
                    \rowcolor{lightblue}    {MONTALVAN MEDINA KAREN ANDREA} &   {211046604} \\
                                            {RIBERA SAAVEDRA FRANZ LEONARDO} &  {210205792} \\
                    \rowcolor{lightblue}    {RODRIGUEZ VALENCIA JOSE LUIS} &    {213115824} \\
                                            {SAGREDO CORDOVA SIMON} &           {215049594} \\
                    \rowcolor{lightblue}    {SOSSA PESOA JUAN ANTONIO} &        {210007664} \\
                    
                    \hline

                    \end{longtable}
            \end{flushleft}
        \end{minipage}
        \hfill  %debes de ajustar el espacio entre la tabla y el QR, no separarlo
        \begin{minipage}{0.20\textwidth}  % 30% del ancho para la imagen QR
            \begin{flushright}
                \includegraphics[width=\linewidth]{\logoqr}  % Cambia "qr_code.png" por tu archivo de imagen QR
            \end{flushright}
        \end{minipage}


        \vspace{1cm}  % Espacio entre el bloque anterior y el siguiente texto
        \Large\textbf{Santa Cruz - Bolivia}\par

    \end{titlepage}
    % ----------------------------FIN CARATULA--------------------------------

    %--------------------------------------------------------------------------
    % Comienzo del INDICE
    \clearpage  % nueva pagina
    \begin{doublespace}     %doble espaciado
    \tableofcontents % Tabla de contenido, se crean automaticamente gracias a las paqueterias
    \end{doublespace}
    \clearpage  %nueva pagina
    %-----------------------------Fin del INDICE-----------------------------


    \hfill
    \vfill
    %  Centrar el texto con letras grandes
    \begin{center}
        \Huge \textbf{SPRINT 0 }\par\vspace{1cm}
        \huge \textbf{preparar el entorno y sentar las bases para el desarrollo del
        proyecto}
    \end{center}
    \vfill
    \hfill




%###########################-- INICIO DEL DOCUMENTO --##############################
\clearpage  %nueva pagina
% Sprint 0
\subfile{Sprint/sprint_0.tex}

\clearpage  %nueva pagina

\hfill
\vfill
%  Centrar el texto con letras grandes
\begin{center}
    \Huge \textbf{SPRINT 1 }\par\vspace{1cm}
    \huge \textbf{Iniciar el desarrollo del proyecto}
\end{center}
\vfill
\hfill


\clearpage  %nueva pagina
% Sprint 1

\subfile{Sprint/sprint_1.tex}


\end{document}
%###########################-- Fin del DOCUMENTO --##############################